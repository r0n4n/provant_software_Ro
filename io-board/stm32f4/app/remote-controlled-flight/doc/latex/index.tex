\hypertarget{index_page_main_sec_overview}{}\section{Overview}\label{index_page_main_sec_overview}
Este é a documentacão do projeto de software {\bfseries Remote Controlled Flight} do {\bfseries pro\+V\+A\+NT}. O objetivo deste projeto de software é gerar um firmware para a {\itshape io-\/board} do V\+A\+NT que permita operá-\/lo com um controle remoto convencional de aeromodelismo. Para tal, são necessárias/previstas o seguinte conjunto de finalidades\+:


\begin{DoxyItemize}
\item Comunicação com controle remoto convecional de modelismo, com leitura de 6 ou mais canais via interface P\+PM.
\item Funcionalidade completa dos E\+S\+Cs e Servo-\/motores do V\+A\+NT.
\item Leitura e tratamento simplificado de dados de uma I\+MU (a ser escolhida).
\item Algoritmo de controle garantindo estabilidade rotacional para o V\+A\+NT durante o vôo.
\end{DoxyItemize}\hypertarget{index_page_main_sec_changelog}{}\section{Changelog}\label{index_page_main_sec_changelog}
$<$$>$Changelog da versão 0.\+3 \mbox{[}Apr-\/2013\mbox{]} \+:
\begin{DoxyItemize}
\item ...
\item ...
\end{DoxyItemize}

{\bfseries Changelog da versão 0.\+2 \mbox{[}Fev-\/2014 $>$ Mar-\/2014\mbox{]} \+:}
\begin{DoxyItemize}
\item Resolvidas as pendências da versão 0.\+1.
\item Implementada a troca de mensagens entre módulos (em threadas).
\item Atuação\+: Implementado os drivers para o servo R\+X24f e o E\+SC da Mikrokopter
\item Sensoriamento\+:
\end{DoxyItemize}

{\bfseries Status da versão 0.\+1 \mbox{[}Nov-\/2013 $>$ Jan-\/2014\mbox{]} \+:}
\begin{DoxyItemize}
\item Implementado o esqueleto básico da estrutura do projeto (sistema de {\bfseries modules} com um {\itshape main} e um {\itshape common})
\item Adotada uma convenção de nomenclatura, descrita em \hyperlink{page_naming}{Organização, nomenclaturas e code-\/style} )
\item Implementadas as funções básicas para\+:
\begin{DoxyItemize}
\item U\+S\+A\+RT (2 e 6; 3 ainda pendente) com tratador de interrupção e buffer circular.
\item I2C (I2\+C1)
\item G\+P\+IO (wrappers) e E\+X\+TI (interrupts externos)
\end{DoxyItemize}
\item Implementados módulos para\+:
\begin{DoxyItemize}
\item Receiver (usando T\+I\+M1 e E\+X\+TI)
\item Servo R\+X24F, portando a biblioteca preexistente do Arduino.
\item I2C (exemplo com I\+MU simples baseada nos C\+Is I\+T\+G3205 e A\+D\+X\+L345)
\end{DoxyItemize}
\item Integração com Free\+R\+T\+OS.
\item Integração e teste com \href{http://www.freertos.org/FreeRTOS-Plus/FreeRTOS_Plus_Trace/FreeRTOS_Plus_Trace.shtml}{\tt Free\+R\+T\+O\+S+\+Trace} e Tracealyzer, ver \hyperlink{page_freertosplustrace}{Usando o Free\+R\+T\+OS + Trace} .
\end{DoxyItemize}

{\bfseries Pendências da versão 0.\+1\+:}
\begin{DoxyItemize}
\item Atualizar a documentação (decritivo da estrutura de projeto, em \hyperlink{page_naming}{Organização, nomenclaturas e code-\/style} )
\item Implementar troca de mensagens entre threads.
\end{DoxyItemize}

Para pendências internas no código, ver lista de tarefas auto-\/gerada (todo ). 