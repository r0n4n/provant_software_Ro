\hypertarget{page_setup_page_setup_sec_introduction}{}\section{Introdução}\label{page_setup_page_setup_sec_introduction}
Esse documento tem como objetivo fornecer os passos necessários para o setup do Open\+O\+CD e do Toolchain de compilacão para targets A\+RM (em especial, para a série Cortex-\/\+M3/\+M4). O {\bfseries Open\+O\+CD} é o programa que permite ao PC programar, gravar, apagar, etc., o A\+RM. Já o {\bfseries Toolchain} é o kit compilador+linker usado para gerar o executável para o A\+RM a partir dos códigos fonte.

Observações\+:
\begin{DoxyItemize}
\item Os comandos nesta seção são previstos para Linux, e foram testados nas distribuições Ubuntu e Linux Mint. Podem existir variações entre distros.
\item O projeto prevê o uso de um A\+RM Cortex-\/\+M4, especificamente a S\+T\+M32\+F407. Alguns comandos deste documento serão específicos para este target.
\item O projeto prevê também o uso de dois tipos de adaptador J\+T\+AG\+: o \href{http://dangerousprototypes.com/docs/Bus_Blaster}{\tt Busblaster}, e o TI I\+C\+DI, embutido na placa \href{http://www.st.com/stm32f4-discovery}{\tt S\+T\+M32\+F4-\/\+Discovery} utilizada. Novamente, alguns comandos deste tutorial são específicos para estes dispositivos.
\end{DoxyItemize}\hypertarget{page_setup_page_setup_sec_introduction_subsec_ton}{}\subsection{Tabela de Nomenclatura}\label{page_setup_page_setup_sec_introduction_subsec_ton}
Ao longo deste documento, alguns termos serão usados com significado bastante específico. Quanto este for o caso, estes termos estarão em {\bfseries negrito}, e seu significado será o listado na tabela abaixo.

\begin{center} \tabulinesep=1mm
\begin{longtabu} spread 0pt [c]{*2{|X[-1]}|}
\hline
\rowcolor{\tableheadbgcolor}{\bf Nomenclatura }&{\bf Uso em contexto  }\\\cline{1-2}
\endfirsthead
\hline
\endfoot
\hline
\rowcolor{\tableheadbgcolor}{\bf Nomenclatura }&{\bf Uso em contexto  }\\\cline{1-2}
\endhead
Adaptador J\+T\+AG&Placa J\+T\+A\+G/\+U\+SB\+: o adaptador {\itshape Bus\+Blaster} é o default do projeto. \\\cline{1-2}
A\+RM &Um {\bfseries target}. No contexto deste documento, é o processador S\+T\+M32\+F407. \\\cline{1-2}
Target &Processador sendo manipulado (programado, debugado, etc.). \\\cline{1-2}
\end{longtabu}
\end{center} \hypertarget{page_setup_page_setup_sec_environmentsetup}{}\section{Setup do Ambiente}\label{page_setup_page_setup_sec_environmentsetup}
\hypertarget{page_setup_page_setup_sec_environmentsetup}{}\subsection{Setup do Ambiente}\label{page_setup_page_setup_sec_environmentsetup}
O Open\+O\+CD (Open on-\/chip Debugger, \mbox{[}2\mbox{]}) é uma ferramenta para debug, gravaçãoao e inspeção de processadores (targets) através de uma ferramenta de interface (J\+T\+AG, I\+SP, etc). Ele será usado para upload e debug da placa, primeiramente pelo terminal, e depois diretamente da I\+DE Eclipse.


\begin{DoxyEnumerate}
\item Instalando as bibliotecas necessárias para compilação; num terminal, digite\+: 
\begin{DoxyCode}
$ sudo apt-\textcolor{keyword}{get} install build-essentials libusb-dev libftdi-dev git
\end{DoxyCode}

\item Para obter a versão mais recente do Open\+O\+CD, clonaremos o repositório. Para tal, no terminal, entre\+: 
\begin{DoxyCode}
$ git clone git:\textcolor{comment}{//repo.or.cz/openocd.git}
\end{DoxyCode}

\item Entre na pasta clonada e execute o comando {\itshape  ./boostrap }. 
\begin{DoxyCode}
$ cd openocd && ./boostrap
\end{DoxyCode}

\item Os módulos a serem compilados no Open\+O\+CD podem ser escolhidos durante a execucução do {\itshape ./configure}. Para configurar os módulos de acordo com o defindo na \hyperlink{page_setup_page_setup_sec_introduction}{Introdução} , execute\+: 
\begin{DoxyCode}
$ ./configure --disable-werror --enable-legacy-ft2232\_libftdi --enable-buspirate --enable-jlink --enable-
      stlink --enable-ti-icdi --enable-usb\_blaster\_libftdi --enable-ulink
\end{DoxyCode}
 É possível que um warning seja emitindo, pois estamos usando uma versão antiga do driver F\+T\+DI\+: 
\begin{DoxyCode}
configure: WARNING: Building the deprecated \textcolor{stringliteral}{'ft2232'} adapter driver.
\end{DoxyCode}
 Este pode ser ignorado sem problemas.
\item Compile o código e instale a aplicação. Execute\+: 
\begin{DoxyCode}
$ make && sudo make install
\end{DoxyCode}

\end{DoxyEnumerate}

Se os passos acima foram realizados sem erros, o comando {\itshape openocd} está disponível em todo o sistema. Pode-\/se verificar no terminal;


\begin{DoxyCode}
$ openocd -h 

Licensed under GNU GPL v2
For bug reports, read
    http:\textcolor{comment}{//openocd.sourceforge.net/doc/doxygen/bugs.html}
Open On-Chip Debugger
Licensed under GNU GPL v2
--help       | -h   display \textcolor{keyword}{this} help
--version    | -v   display OpenOCD version
--file       | -f   use configuration file <name>
--search     | -s   dir to search \textcolor{keywordflow}{for} config files and scripts
--debug      | -d   \textcolor{keyword}{set} debug level <0-3>
--log\_output | -l   redirect log output to file <name>
--command    | -c   run <command>
\end{DoxyCode}
\hypertarget{page_setup_page_setup_sec_launchopenocd}{}\subsection{Lançando o Open\+O\+C\+D e conectando à placa}\label{page_setup_page_setup_sec_launchopenocd}
Os comandos do Open\+O\+CD costumam assumir a forma\+: 
\begin{DoxyCode}
$ openocd -f [script de interface] -f [script de target]
\end{DoxyCode}


Assumiremos que o Open\+O\+CD foi instalado no diretório default\+: {\itshape  /usr/local/shared/openocd }

{\bfseries  Instruções para a placa S\+T\+M32\+F4 H407 }


\begin{DoxyEnumerate}
\item Conecte a placa à uma fonte de alimentação (6 ou 16V), ou à uma porta U\+SB (dependendo da configuração do jumper {\itshape P\+W\+R\+\_\+\+S\+EL}).
\item Conecte o {\bfseries adaptador J\+T\+AG} à porta U\+SB.
\item Conecte a placa ao adaptador através de um cabo flat de 20 vias.
\item Num terminal, entre\+: 
\begin{DoxyCode}
openocd -f /usr/local/share/openocd/scripts/interface/busblaster.cfg -f /usr/local/share/openocd/scripts/
      target/stm32f4x.cfg
\end{DoxyCode}

\end{DoxyEnumerate}

{\bfseries  Instruções para a placa S\+T\+M32\+F4 H407 }

dolor sit amet 